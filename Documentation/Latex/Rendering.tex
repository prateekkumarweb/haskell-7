\haddockmoduleheading{Rendering}
\label{module:Rendering}
\haddockbeginheader
{\haddockverb\begin{verbatim}
module Rendering (
    player1Color,  player2Color,  gridLightPlayer1,  gridLightPlayer2, 
    blackcolor,  whitecolor,  tieColor,  boardGrid,  menuBoardLine, 
    rectangleMenuBlack,  tileRenderMulti,  tileRenderSingle,  boardMenuPicture, 
    boardAsRunningPicture,  snapPictureToCell,  snapPictureToMenuCell,  aCell, 
    bCell,  dotCell,  cellsOfBoard,  menuCellsBoard,  tileRenderMultiCell, 
    tileRenderSingleCell,  aCellsOfBoard,  bCellsOfBoard,  dotCellsOfBoard, 
    gameAsPicture
  ) where\end{verbatim}}
\haddockendheader

\begin{haddockdesc}
\item[\begin{tabular}{@{}l}
player1Color\ ::\ Color
\end{tabular}]\haddockbegindoc
Intial colors for the different graphics\par

\end{haddockdesc}
\begin{haddockdesc}
\item[
player2Color\ ::\ Color
]
\item[
gridLightPlayer1\ ::\ Color
]
\item[
gridLightPlayer2\ ::\ Color
]
\item[
blackcolor\ ::\ Color
]
\item[
whitecolor\ ::\ Color
]
\item[
tieColor\ ::\ Color
]
\end{haddockdesc}
\begin{haddockdesc}
\item[\begin{tabular}{@{}l}
boardGrid\ ::\ Picture
\end{tabular}]\haddockbegindoc
\haddockid{boardGrid} defines the lines on the board\par

\end{haddockdesc}
\begin{haddockdesc}
\item[\begin{tabular}{@{}l}
menuBoardLine\ ::\ Picture
\end{tabular}]\haddockbegindoc
\haddockid{menuBoardLine} defines the lines on the menu board\par

\end{haddockdesc}
\begin{haddockdesc}
\item[
rectangleMenuBlack\ ::\ Picture
]
\end{haddockdesc}
\begin{haddockdesc}
\item[\begin{tabular}{@{}l}
tileRenderMulti\ ::\ Picture
\end{tabular}]\haddockbegindoc
\haddockid{tileRenderMulti} returns the Picture from the string for Multi cell part\par

\end{haddockdesc}
\begin{haddockdesc}
\item[\begin{tabular}{@{}l}
tileRenderSingle\ ::\ Picture
\end{tabular}]\haddockbegindoc
\haddockid{tileRenderSingle} returns the Picture from the string for Single cell part\par

\end{haddockdesc}
\begin{haddockdesc}
\item[
boardMenuPicture\ ::\ Game\ ->\ MenuBoard\ ->\ Picture
]
\end{haddockdesc}
\begin{haddockdesc}
\item[\begin{tabular}{@{}l}
boardAsRunningPicture\ ::\ Game\ ->\ Board\ ->\ Picture
\end{tabular}]\haddockbegindoc
\haddockid{boardAsRunningPicture} renders the picture depending on the board provided\par

\end{haddockdesc}
\begin{haddockdesc}
\item[\begin{tabular}{@{}l}
snapPictureToCell\ ::\ Picture\ ->\ (Int,\ Int)\ ->\ Picture
\end{tabular}]\haddockbegindoc
\haddockid{snapPictureToCell} places the picture back to the cell\par

\end{haddockdesc}
\begin{haddockdesc}
\item[
snapPictureToMenuCell\ ::\ Picture\ ->\ (Float,\ Float)\ ->\ Picture
]
\end{haddockdesc}
\begin{haddockdesc}
\item[\begin{tabular}{@{}l}
aCell\ ::\ Picture
\end{tabular}]\haddockbegindoc
\haddockid{aCell} returns a solid circle picture with a defined radius of Player 1\par

\end{haddockdesc}
\begin{haddockdesc}
\item[\begin{tabular}{@{}l}
bCell\ ::\ Picture
\end{tabular}]\haddockbegindoc
\haddockid{bCell} returns a solid circle picture with a defined radius of Player 2\par

\end{haddockdesc}
\begin{haddockdesc}
\item[\begin{tabular}{@{}l}
dotCell\ ::\ Picture
\end{tabular}]\haddockbegindoc
\haddockid{dotCell} returns a solid circle picture with a smaller defined radius for no players\par

\end{haddockdesc}
\begin{haddockdesc}
\item[\begin{tabular}{@{}l}
cellsOfBoard\ ::\ Board\ ->\ Cell\ ->\ Picture\ ->\ Picture
\end{tabular}]\haddockbegindoc
\haddockid{cellsOfBoard} maps the  picture to the cells according to the cell type\par

\end{haddockdesc}
\begin{haddockdesc}
\item[\begin{tabular}{@{}l}
menuCellsBoard\ ::\ MenuBoard\ ->\ Choice\ ->\ Picture\ ->\ Picture
\end{tabular}]\haddockbegindoc
\haddockid{menuCellsBoard} is in accordance to the \haddockid{cellsOfBoard} function\par

\end{haddockdesc}
\begin{haddockdesc}
\item[
tileRenderMultiCell\ ::\ MenuBoard\ ->\ Picture
]
\item[
tileRenderSingleCell\ ::\ MenuBoard\ ->\ Picture
]
\end{haddockdesc}
\begin{haddockdesc}
\item[\begin{tabular}{@{}l}
aCellsOfBoard\ ::\ Board\ ->\ Picture
\end{tabular}]\haddockbegindoc
\haddockid{aCellsOfBoard} returns the cells where the cells are of \haddocktt{a} type (Full Player1)\par

\end{haddockdesc}
\begin{haddockdesc}
\item[\begin{tabular}{@{}l}
bCellsOfBoard\ ::\ Board\ ->\ Picture
\end{tabular}]\haddockbegindoc
\haddockid{bCellsOfBoard} returns the cells where the cells are of \haddocktt{b} type (Full Player2)\par

\end{haddockdesc}
\begin{haddockdesc}
\item[\begin{tabular}{@{}l}
dotCellsOfBoard\ ::\ Board\ ->\ Picture
\end{tabular}]\haddockbegindoc
\haddockid{dotCellsOfBoard} returns the cells where the cells are of \haddocktt{dot} type (Full Dot)\par

\end{haddockdesc}
\begin{haddockdesc}
\item[\begin{tabular}{@{}l}
gameAsPicture\ ::\ Game\ ->\ Picture
\end{tabular}]\haddockbegindoc
\haddockid{gameAsPicture} renders the board or the menu screen depending on the Game state\par

\end{haddockdesc}